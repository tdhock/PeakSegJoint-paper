\documentclass{article} % For LaTeX2e
\usepackage{nips15submit_e,times}
\usepackage{tikz}
\usepackage{algorithm}
\usepackage{algorithmic}
\usepackage{hyperref}
\usepackage{url}
\usepackage{natbib}
\usepackage{graphicx}
\usepackage{amsmath,amssymb,amsthm}
\newtheorem{proposition}{Proposition}
\DeclareMathOperator*{\argmin}{arg\,min}
\DeclareMathOperator*{\Lik}{Lik}
\DeclareMathOperator*{\Peaks}{Peaks}
\DeclareMathOperator*{\Segments}{Segments}
\DeclareMathOperator*{\argmax}{arg\,max}
\DeclareMathOperator*{\maximize}{maximize}
\DeclareMathOperator*{\minimize}{minimize}
\newcommand{\sign}{\operatorname{sign}}
\newcommand{\RR}{\mathbb R}
\newcommand{\ZZ}{\mathbb Z}
\newcommand{\NN}{\mathbb N}
\definecolor{noPeaks}{HTML}{F6F4BF}
\definecolor{peakStart}{HTML}{FFAFAF}
\definecolor{peakEnd}{HTML}{FF4C4C}
\definecolor{peaks}{HTML}{A445EE}
\newcommand{\JointHeuristic}{\textsc{JointZoom}}

\title{PeakSegJoint: fast supervised peak detection via joint
  segmentation of multiple count data samples}


\author{
David S.~Hippocampus\thanks{ Use footnote for providing further information
about author (webpage, alternative address)---\emph{not} for acknowledging
funding agencies.} \\
Department of Computer Science\\
Cranberry-Lemon University\\
Pittsburgh, PA 15213 \\
\texttt{hippo@cs.cranberry-lemon.edu} \\
\And
Coauthor \\
Affiliation \\
Address \\
\texttt{email} \\
\AND
Coauthor \\
Affiliation \\
Address \\
\texttt{email} \\
\And
Coauthor \\
Affiliation \\
Address \\
\texttt{email} \\
\And
Coauthor \\
Affiliation \\
Address \\
\texttt{email} \\
(if needed)\\
}

% The \author macro works with any number of authors. There are two commands
% used to separate the names and addresses of multiple authors: \And and \AND.
%
% Using \And between authors leaves it to \LaTeX{} to determine where to break
% the lines. Using \AND forces a linebreak at that point. So, if \LaTeX{}
% puts 3 of 4 authors names on the first line, and the last on the second
% line, try using \AND instead of \And before the third author name.

\newcommand{\fix}{\marginpar{FIX}}
\newcommand{\new}{\marginpar{NEW}}

%\nipsfinalcopy % Uncomment for camera-ready version

\begin{document}


\maketitle

\begin{abstract}
  Peak detection algorithms identify enriched regions in genomic count
  data. Joint peak detectors attempt to find common peaks in several
  samples, but current algorithms for this task are unsupervised and
  limited to at most 2 sample types. We propose PeakSegJoint, a new
  constrained maximum likelihood segmentation model for any number of
  sample types. To select the number of peaks in the segmentation, we
  propose a supervised penalty learning model. To infer the parameters
  of these two models, we propose a discrete optimization heuristic
  for the segmentation, and a gradient-based convex optimization
  algorithm for the penalty learning. In comparisons with
  state-of-the-art peak detection algorithms, PeakSegJoint achieves
  similar accuracy, faster speeds, and a more interpretable model with
  overlapping peaks that occur in exactly the same positions across
  all samples.
\end{abstract}

\section{Introduction: Joint supervised peak detection in ChIP-seq data}

%\subsection{}

Chromatin immunoprecipitation sequencing (ChIP-seq) is an experimental
technique used for genome-wide profiling of histone modifications and
transcription factor binding sites \citep{practical}. Each experiment
yields a set of sequence reads which are aligned to a reference
genome, and then the number of aligned reads are counted at each
genomic position. To compare samples at a given genomic position,
biologists visually examine coverage plots such as
Figure~\ref{fig:good-bad} for presence or absence of common ``peaks.''
In machine learning terms, a single sample over $B$ base positions is
a vector of non-negative count data $\mathbf z\in\ZZ_+^B$ and a peak
detector is a binary classifier $c:\ZZ_+^B\rightarrow\{0, 1\}^B$. The
positive class is peaks and the negative class is background
noise. Importantly, peaks and background occur in long contiguous
segments across the genome.

In this paper we use the supervised learning framework of
\citet{hocking2014visual}, who proposed to use labels obtained from
manual visual inspection of genome plots. For each labeled genomic
region $i\in\{1, \dots, n\}$ there is a set of count data $\mathbf
z_i$ and labels $L_i$ (``noPeaks,'' ``peaks,'' etc. as in
Figure~\ref{fig:good-bad}). These labels define a non-convex
annotation error function
\begin{equation}
  \label{eq:error}
  E[c(\mathbf z_i),  L_i] =
  \text{FP}[c(\mathbf z_i), L_i] +
  \text{FN}[c(\mathbf z_i), L_i]
\end{equation}
which counts the number of false positive (FP) and false negative (FN)
regions, so it takes values in the non-negative integers. In this
framework the goal of learning is to find a peak detection algorithm
$c$ that minimizes the number of incorrect labels (\ref{eq:error}) on
a test set of data:
\begin{equation}
  \label{eq:min_error}
  \minimize_c \sum_{i\in\text{test}} E[c(\mathbf z_i),  L_i].
\end{equation}

\begin{figure}[b!]
  \centering
  \includegraphics[width=\textwidth]{figure-good-bad}
  \vskip -0.5cm
  \caption{Labeled ChIP-seq coverage data for $S=4$ samples (top 4
    panels) with 3 peak models (bottom 3 panels), which predict peaks 
    in blue regions and background noise in other regions.
    % \colorbox{noPeaks}{noPeaks} means there should be no overlapping
    % peaks, \colorbox{peakStart}{peakStart}/\colorbox{peakEnd}{peakEnd}
    % mean there should be exactly 1 peak start/end somewhere in the
    % region, and \colorbox{peaks}{peaks} means that ) 
    An ideal peak model minimizes the number of incorrect labels
    (false positives are too many peaks and false negatives are not
    enough peaks). The ``good'' model achieves 0 errors but the
    ``better'' model is more interpretable since overlapping peaks in
    different samples occur in the exact same positions.}
  \label{fig:good-bad}
\end{figure}

In practical situations we have $S>1$ samples, and a matrix $\mathbf
Z\in\ZZ_+^{B\times S}$ of count data. For example in
Figure~\ref{fig:good-bad} we have $S=4$ samples. In these data we are
not only interested in accurate peak detection in individual samples,
but also in detecting the differences between samples. In particular,
an ideal model of a multi-sample data set is a \textbf{joint} peak
detector with identical overlapping peak positions across samples (the
``better'' model in Figure~\ref{fig:good-bad}).

\subsection{Contributions and organization}

The main contribution of this paper is the \ref{PeakSegJoint} model
for supervised joint peak detection. Unlike previous methods
(Section~\ref{sec:related}), it is the first joint peak detector that
explicitly models any number of sample types
(Section~\ref{sec:models}). A secondary contribution is the
\JointHeuristic\ heuristic segmentation algorithm, and a supervised
learning algorithm for efficiently selecting the \ref{PeakSegJoint}
model complexity (Section~\ref{sec:algorithms}).  We show test error
and timing results in Section~\ref{sec:results}, provide a discussion
in Section~\ref{sec:discussion}, and propose some future research
directions in Section~\ref{sec:conclusions}.

\section{Related work}
\label{sec:related}

\subsection{Single-sample ChIP-seq peak detectors}

There are many different unsupervised peak detection algorithms
\citep{evaluation2010, rye2010manually, chip-seq-bench}, and the
current state-of-the-art is the supervised \ref{PeakSeg} model of
\citet{HOCKING-PeakSeg}. 

In the multi-sample setting of this paper, each of these algorithms
may be applied independently to each sample. The drawback of these
methods is that the peaks do not occur in the same positions across
each sample, so it is not straightforward to determine differences
between samples.

\subsection{Methods for several data sets}

This paper is concerned with the particular setting of analyzing
several samples (perhaps of different cell types) of the same
experiment data type. For example, Figure~\ref{fig:good-bad} shows
H3K4me3 experimental data for 2 bcell samples, 1 monocyte sample, and
1 tcell sample (different types of blood cells). As far as we know,
there are no existing algorithms that can explicitly model these data.

For these data the most applicable peak detectors in the
bioinformatics literature model several samples of the same cell
type. For example, the JAMM algorithm of \citet{JAMM} can analyze
several samples, but is limited to a single cell type since it assumes
that each sample is a replicate with the exact same peak pattern. So
to analyze the data of Figure~\ref{fig:good-bad} one would have to run
JAMM 3 times (once for each cell type), and the resulting peaks would
not be the same across cell types. Another example is the PePr
algorithm of \citet{PePr}, which can model either one or two cell
types, but is unsuitable for analysis of three or more cell types. In
contrast, \ref{PeakSegJoint} is for data from several samples without
limit on the number of cell types.

Although not the subject of this paper, there are several algorithms
designed for the analysis of data from several different ChIP-seq
experiments \citep{chromhmm,segway,jmosaics}. In these models, the
input is one sample (e.g. monocyte cells) with different experiments
such as H3K4me3 and H3K36me3. Also,
\citet{hierarchical-joint} proposed a model for one sample with both
ChIP-seq and ChIP-chip data. In contrast, 
\ref{PeakSegJoint} takes as input several samples
(e.g. monocyte, tcell, bcell) for one experiment such as H3K4me3.

\section{Models}\label{sec:models}

We begin by summarizing the single-sample \ref{PeakSeg} model, and
then introduce the multi-sample \ref{PeakSegJoint} model.

\subsection{PeakSeg: finding the most likely $0,\dots,p_{\text{max}}$
  peaks in a single sample}

This section describes the \ref{PeakSeg} model of
\citet{HOCKING-PeakSeg}, which is the current state-of-the-art peak
detection algorithm in the benchmark data set of
\citet{hocking2014visual}. 

Given a single sample profile $\mathbf z\in\ZZ_+^B$ of aligned read
counts on $B$ bases, the PeakSeg model for the mean vector $\mathbf
m\in\RR^B$ is $z_j\sim \text{Poisson}(m_j)$. For a maximum number of
peaks $p_{\text{max}}$ the constrained maximum likelihood PeakSeg
model with $p\in\{0, \dots, p_{\text{max}}\}$ peaks is defined as
\begin{align}
  \label{PeakSeg}
  \mathbf{\tilde m}^p(\mathbf z)  =
    \argmin_{\mathbf m\in\RR^{B}} &\ \ 
    \text{PoissonLoss}(\mathbf m, \mathbf z) 
    \tag{\textbf{PeakSeg}}
\\
    \text{such that} &\ \  \Peaks(\mathbf m)=p, \label{peaks=p} \\
     \forall j\in\{1, \dots, B\}, &\ \ P_j(\mathbf m) \in\{0, 1\},
    \label{eq:peak_constraint}
\end{align}
where the Poisson loss function is
\begin{equation}\label{eq:rho}
  \text{PoissonLoss}(\mathbf m, \mathbf z)= \sum_{j=1}^B m_j - z_j \log m_j.
\end{equation} 
Note that the optimization objective of minimizing the Poisson loss is
equivalent to maximizing the Poisson likelihood. The model complexity
(\ref{peaks=p}) is the number of peaks
\begin{equation}
  \Peaks(\mathbf m)=(\Segments(\mathbf m)-1)/2,
\end{equation}
which is a function of the number of piecewise constant segments
\begin{equation}
  \Segments(\mathbf m)=1+\sum_{j=2}^B I(m_j \neq m_{j-1}).
\end{equation}
Finally, the peak indicator at base $j$ is defined as the cumulative sum of
signs of changes
\begin{equation}
  \label{eq:peaks}
  P_j(\mathbf m) = \sum_{k=2}^j \sign( m_{k} - m_{k-1} ),
\end{equation}
with $P_1(\mathbf m)=0$ by convention. The constraint
(\ref{eq:peak_constraint}) means that the peak indicator $P_j(\mathbf
m)\in\{0, 1\}$ can be used to classify each base $j\in\{1,\dots B\}$
as either background noise $P_j(\mathbf m)=0$ or a peak $P_j(\mathbf
m)=1$.  In other words, $\mathbf{\tilde m}^p(\mathbf z)$ is a
piecewise constant vector that changes up, down, up, down (and not up,
up, down). Thus the even numbered segments are interpreted as peaks,
and the odd numbered segments are interpreted as background noise.

Note that since \ref{PeakSeg} is defined for a single sample, it may
be independently applied to each sample in data sets such as
Figure~\ref{fig:good-bad}. However, any overlapping peaks in different
samples will not necessarily occur in the same positions. In the next
section we fix this problem by proposing the more interpretable
multi-sample \ref{PeakSegJoint} model.

% \subsection{A multi-sample model 
%   related to the single-sample PeakSeg
%   model}

% For $S$ sample profiles $\mathbf z_1, \dots, \mathbf z_S\in\ZZ_+^B$
% defined on the same $B$ bases, we stack the vectors into a matrix
% $\mathbf Z\in\ZZ_+^{B \times S}$. Consider the following model for the
% mean matrix $\mathbf M\in\RR^{B\times S}$ which allows each sample to
% have either 0 or 1 peaks, with a total of $p\in\{0, \dots, S\}$ peaks:
% \begin{align}
%   \label{Unconstrained}
%   \mathbf{\tilde M}^p(\mathbf Z)  =
%   \argmin_{\mathbf M\in\RR^{B\times S}} &\ \ 
%   \sum_{s=1}^S 
%   \text{PoissonLoss}(\mathbf m_s, \mathbf z_s) 
%   \tag{\textbf{Unconstrained}}
%   \\
%   \text{such that} &\ p = \sum_{s=1}^S \Peaks(\mathbf m_s)
%   \label{total_Peaks}
%   \\
%   &\ \forall s\in\{1, \dots, S\},\, 
%   \Peaks(\mathbf m_s)\in\{0, 1\},  
%   \label{zero_or_one}
%   \\
%   &\ \forall s\in\{1, \dots, S\},\, 
%   \forall j\in\{1, \dots, B\},\, P_j(\mathbf m_s) \in\{0, 1\}.
%   \label{up_down}
% \end{align}
% This optimization problem is \ref{Unconstrained} in the sense that the
% peaks are not required to be in the exact same locations in each of
% the $S$ samples. The constraint (\ref{up_down}) is the same constraint
% as in \ref{PeakSeg}, which here requires the segment mean $\mathbf
% m_s$ of each sample to have alternating changes (up, down, up, down
% and not up, up, down). The constraint on the number of peaks per
% sample (\ref{zero_or_one}) means that each sample may have either zero
% or one peak. Finally, the overall constraint (\ref{total_Peaks}) means
% that there is a total of $p$ samples each with exactly 1 peak (not
% necessarily with the same start/end positions across samples, see
% Figure~\ref{fig:PeakSegJoint}).

% \begin{proposition}
% The \ref{Unconstrained} solution in $p$ peaks
% can be written in terms of the \ref{PeakSeg} solutions in 0 or 1 peaks:
% \begin{equation}
%   \label{eq:unconstrained_PeakSeg}
%   \mathbf{\tilde M}^p(\mathbf Z) = \left[
%     \begin{array}{ccc}
%       \mathbf{\tilde m}^{p_1}(\mathbf z_1) & 
%       \cdots &
%       \mathbf{\tilde m}^{p_S}(\mathbf z_S) 
%     \end{array}
%   \right],
% \end{equation}
% for some $p_1,\dots, p_S\in\{0, 1\}$ with $p=\sum_{i=s}^S p_s$.
% \end{proposition}

% \begin{proof}
%   Since the objective function of \ref{Unconstrained} is separable on
%   samples $s$, it can be re-written as $p$ \ref{PeakSeg} sub-problems
%   each with 1 peak, and $S-p$ trivial sub-problems with 0 peaks.
%   % To begin, consider the trivial case with $p=0$ peaks. Clearly, the
%   % PeakSegJoint solution is
%   % $\mathbf{\tilde M}^p(\mathbf Z) = \left[
%   %   \begin{array}{ccc}
%   %     \mathbf{\bar z}_1 & 
%   %     \cdots &
%   %     \mathbf{\bar z}_S 
%   %   \end{array}
%   % \right]$, where $\mathbf{\bar z}_s$ is the constant mean vector for
%   % sample $s$. This is also the PeakSeg model with 0 peaks $\mathbf{\tilde
%   %   m}^0(\mathbf z_s)=\mathbf{\bar z}_s$. 
  
%   % Now consider the case for $p=1$ peak. There are $S$ possible models,
%   % one for each sample $s$. The sample with a peak model
%   % $\mathbf{\tilde m}^1(\mathbf z_{s^*})$ is $s^*=\argmin_{s\in
%   %   1,\dots, S} \textrm{PoissonLoss}\left[ \mathbf{\tilde m}^1(\mathbf
%   %   z_s), \mathbf z_s \right] - \textrm{PoissonLoss}\left[
%   %   \mathbf{\tilde m}^0(\mathbf z_s), \mathbf z_s \right]$, and all
%   % the other samples $s$ have 0 peaks $\mathbf{\tilde m}^0(\mathbf
%   % z_{s})$. This model is clearly optimal since the \ref{Unconstrained} cost
%   % function is separable on samples.

%   % Consider the case for $p=2$ peaks. By the constraint of either zero
%   % or one peak for each sample (\ref{zero_or_one}), and the constraint
%   % on the total number of peaks (\ref{total_Peaks}), it is clear that
%   % the optimal solution has $p=2$ samples with exactly 1 peak and $S-2$
%   % samples with exactly 0 peaks. The samples with peaks are $s^*$ and
%   % $\argmin_{s\neq s^*} \textrm{PoissonLoss}\left[ \mathbf{\tilde
%   %     m}^1(\mathbf z_s), \mathbf z_s \right] -
%   % \textrm{PoissonLoss}\left[ \mathbf{\tilde m}^0(\mathbf z_s), \mathbf
%   %   z_s \right]$.

%   % The proof for $p>2$ peaks proceeds by induction.
% \end{proof}

\subsection{PeakSegJoint: finding the most likely common peak in
  $0,\dots, S$ samples}

\begin{figure}[b!]
  \centering
  \includegraphics[width=\textwidth]{figure-PeakSegJoint}
  \vskip -0.5cm
  \caption{A labeled data set with $S=3$ samples (top 3 panels) and
    the \ref{PeakSegJoint} models for $p\in\{0, 1, 2, 3\}$ peaks
    (bottom 4 panels). Blue positions show model peaks and other positions 
    show background noise.
%The    ``Unconstrained'' model does not include the \ref{PeakSegJoint}    constraint (\ref{joint_constraint}) that overlapping peaks must    occur in exactly the same positions.
  }
  \label{fig:PeakSegJoint}
\end{figure}

For $S$ sample profiles $\mathbf z_1, \dots, \mathbf z_S\in\ZZ_+^B$
defined on the same $B$ bases, we stack the vectors into a matrix
$\mathbf Z\in\ZZ_+^{B \times S}$ of count data. Consider the following
model for the mean matrix $\mathbf M\in\RR^{B\times S}$ which allows
each sample to have either 0 or 1 peaks, with a total of $p\in\{0,
\dots, S\}$ peaks:
% The \ref{PeakSegJoint} model with $p\in\{0, \dots, S\}$ peaks is
% defined using the same Poisson loss function as the \ref{PeakSeg}
% single-sample model, and adding some other constraints:
\begin{align}
  \label{PeakSegJoint}
  \mathbf{\hat M}^p(\mathbf Z)  =
  \argmin_{\mathbf M\in\RR^{B\times S}} &\ \ 
  \sum_{s=1}^S 
  \text{PoissonLoss}(\mathbf m_s, \mathbf z_s) 
  \tag{\textbf{PeakSegJoint}}
  \\
  \text{such that} &\ 
  \forall s\in\{1, \dots, S\},\, 
  \Peaks(\mathbf m_s)\in\{0, 1\},  
  \label{zero_or_one}
  \\
  &\ 
  \forall s\in\{1, \dots, S\},\,
  \forall j\in\{1, \dots, B\},\, P_j(\mathbf m_s) \in\{0, 1\},
  \label{joint_up_down}
  \\
  &\ 
  p = \sum_{s=1}^S \Peaks(\mathbf m_s),
  \label{total_peaks}
  \\
  &\ \forall s_1\neq s_2\mid
  \nonumber
  \Peaks(\mathbf m_{s_1})=\Peaks(\mathbf  m_{s_2})=1,\,
  \forall j\in\{1, \dots, B\},\\
  &\ \ P_j(\mathbf m_{s_1}) = P_j(\mathbf m_{s_2}).
  \label{joint_constraint}
\end{align}
The first two constraints are similar to the \ref{PeakSeg} model
constraints. The constraint on the number of peaks per sample
(\ref{zero_or_one}) means that each sample may have either zero or one
peak. The constraint (\ref{joint_up_down}) requires the segment mean
$\mathbf m_s$ of each sample to have alternating changes (up, down,
up, down and not up, up, down). The overall constraint
(\ref{total_peaks}) means that there is a total of $p$ samples with
exactly 1 peak. The last constraint (\ref{joint_constraint}) means
that peaks should occur in the exact same positions in each sample.
The \ref{PeakSegJoint} models for a data set with $S=3$ samples is
shown in Figure~\ref{fig:PeakSegJoint}.

\subsection{Supervised penalty learning}

In the last section we considered data $\mathbf Z\in\ZZ_+^{B\times S}$
for $S$ samples in a single genomic region with $B$ bases. Now assume
that we have data $\mathbf Z_1,\dots, \mathbf Z_n\in\ZZ_+^{B\times S}$
for $n$ genomic regions, along with annotated region labels
$L_1,\dots, L_n$. 

Faor each genomic region $i\in\{1,\dots,n\}$ we can compute a sequence
of \ref{PeakSegJoint} models $\mathbf{\hat M}^0(\mathbf Z_i),\dots,
\mathbf{\hat M}^S(\mathbf Z_i)$, but how will we predict which of
these $S+1$ models will be best?
% in terms of the test annotated region labels $L_i$? 
This is the segmentation model selection problem, which we propose to
solve via supervised learning of a penalty function.

First, for a positive penalty constant $\lambda\in\RR_+$, we define
the optimal number of peaks as
\begin{equation}
  \label{eq:optimal_segments}
  p^*(\lambda, \mathbf Z) =
  \argmin_{p\in\{0, \dots, S\}}
  p \lambda + 
  \text{PoissonLoss}\left[
    \mathbf{\hat M}^p(\mathbf Z),
    \mathbf Z
  \right],
\end{equation}
The penalty $\lambda$ controls the tradeoff between minimizing the
Poisson loss (a decreasing function of $p$) and the model complexity
($p$).  Also suppose that we can compute $d$-dimensional
sample-specific feature vectors $\mathbf x\in\RR^d$ (mean, quantiles,
etc) and stack them to obtain feature matrices $\mathbf X_1,\dots,
\mathbf X_n\in\RR^{d\times S}$. We will learn a function
$f:\RR^{d\times S}\rightarrow\RR$ that predicts region-specific
penalty values $f(\mathbf X_i) = \log \lambda_i\in\RR$ for each
genomic region $i$. In particular we will learn a weight vector
$\mathbf w\in\RR^d$ in a linear function $f_{\mathbf w}(\mathbf X) =
\mathbf w^\intercal \mathbf X \mathbf 1_S$, where $\mathbf 1_S$ is a
vector of $S$ ones.

For supervision we use the annotated region labels $L_i$ to compute
the number of incorrect regions (\ref{eq:error}) for each model size
$p$, and then a target interval $\mathbf y_i = ( \underline y_i,
\overline y_i )$ of penalty values \citep{HOCKING-penalties}.
Briefly, a predicted penalty in the target interval $f(\mathbf
X_i)\in\mathbf y_i$ implies that the \ref{PeakSegJoint} model with
$p^*\left[\exp f(\mathbf X_i), \mathbf Z_i\right]$ peaks achieves the
minimum number of incorrect labels $L_i$, among all $S+1$
\ref{PeakSegJoint} models for genomic region $i$.
%  $\exp f(\mathbf X_i)\in \argmin_\lambda
% E\left[ \mathbf P\big( \mathbf{\hat M}^{p^*(\lambda, \mathbf
%     Z)}(\mathbf Z) \big), L_i\right]$ (TODO: should we actually show
% this equation or just use words?).

A target interval $\mathbf y$ is used with the squared hinge loss
$\phi(x)=(x-1)^2 I(x\leq 1)$ to define the surrogate loss
\begin{equation}
  \label{eq:surrogate_loss}
  \ell\left[
    \mathbf y,\,
    \log \hat \lambda
    \right]
    =
    \phi\big[
      \log\hat\lambda - \underline y
    \big]
    +
    \phi\big[
    \overline y - \log\hat\lambda
    \big],
\end{equation}
for a predicted penalty value $\hat \lambda\in\RR_+$. For a weight parameter
$\mathbf w\in\RR^d$, the convex average surrogate loss is
\begin{equation}
  \label{eq:average_surrogate}
  \mathcal L(\mathbf w) =
  \frac 1 n
  \sum_{i=1}^n
  \ell\left[
    \mathbf y_i,\,
     f_{\mathbf w}( \mathbf X_i )
    \right].
\end{equation}
Finally, we add a 1-norm penalty to regularize and encourage a sparse
weight vector, thus obtaining the following convex supervised penalty
learning problem:
\begin{equation}
  \label{argmin_w}
  \mathbf{\hat w}^\gamma = 
  \argmin_{\mathbf w\in\RR^d}
  \mathcal L(\mathbf w) + \gamma ||\mathbf w||_1.
\end{equation}

To predict on test data $\mathbf Z$ with features $\mathbf X$, we
compute the predicted penalty $\hat \lambda = \exp f_{\mathbf{\hat
    w}}(\mathbf X)$, the predicted number of peaks $\hat p = p^*(\hat
\lambda, \mathbf Z)$, and finally the predicted mean matrix
$\mathbf{\hat M}^{\hat p}(\mathbf Z)$. Each column/sample of the mean
matrix has either two changes (the second segment is the peak) or no
changes (no peak).

\section{Algorithms}
\label{sec:algorithms}

\subsection{Heuristic discrete optimization for joint segmentation}

The \ref{PeakSegJoint} model is defined as the solution to an
optimization problem with a convex objective function and non-convex
constraints.  Real data sets $\mathbf Z\in\ZZ_+^{B\times S}$ may have
a very large number of data points to segment $B$. Explicitly
computing the maximum likelihood and feasibility for all $O(B^2)$
possible peak start/endpoints is guaranteed to find the global
optimum, but would take too much time.  Instead, we propose to find an
approximate solution using a new discrete optimization algorithm
called \JointHeuristic~(Algorithm~1).

The main idea of the \JointHeuristic\ algorithm is to first zoom out
(downsample the data) repeatedly by a factor of $\beta$, obtaining a
new data matrix of size $b\times S$, where $b \ll B$
(line~\ref{zoomout}). Then we solve the \ref{PeakSegJoint} problem for
$p$ peaks via \textsc{GridSearch} (line~\ref{gridsearch}), a
sub-routine that checks all $O(b^2)$ possible peak start and end
positions. Then we zoom in by a factor of $\beta$ (line~\ref{zoomin})
and refine the peak positions in $O(\beta^2)$ time via
\textsc{SearchNearPeak} (line~\ref{searchnear}). After having zoomed
back in to the BinSize=1 level we return the final Peak
positions. 

\begin{figure}[b!]
  \centering
  \includegraphics[width=0.9\textwidth]{figure-heuristic-algo}
  \vskip -0.5cm
  \caption{Demonstration of the \JointHeuristic\ segmentation
    algorithm. For a data set with $B=24$ data points and $S=2$
    samples (top 2 panels), the algorithm with zoom factor of
    $\beta=2$ proceeds as follows. First, the Poisson loss and
    feasibility (\ref{joint_up_down}) is computed via
    \textsc{GridSearch} over all peak starts and ends at bin size
    4. The feasible model with minimum Poisson loss is selected, the
    bin size is decreased to 2, and \textsc{SearchNearPeak} considers
    a new set of models around the selected peak starts and ends. We
    continue and return the optimal model at bin size 1 (shown in
    green). Interactive figure at \url{http://bit.ly/1AA6TgK}}
  \label{fig:heuristic-algo}
\end{figure}

\begin{algorithm}[H]
\begin{algorithmic}[1]
  \REQUIRE count data $\mathbf Z\in\ZZ_+^{B\times S}$, number of
  peaks $p\in\{0, \dots, S\}$, zoom factor
  $\beta\in\{2, 3, \dots\}$.
  \STATE $\textrm{BinSize} \gets \textsc{MaxBinSize}(B, \beta)$. \label{zoomout}
  \STATE $\textrm{Peak}, \textrm{Samples} \gets \label{gridsearch}
  \textsc{GridSearch}(\mathbf Z, p, \textrm{BinSize})$.
  \WHILE{$1 < \textrm{BinSize}$}
  \STATE $\textrm{BinSize} \gets \textrm{BinSize} / \beta$. \label{zoomin}
  \STATE $\textrm{Peak} \gets
  \textsc{SearchNearPeak}(\mathbf Z, \textrm{Samples}, \label{searchnear}
  \textrm{BinSize}, \textrm{Peak})$
  \ENDWHILE
  \RETURN Peak, Samples.
  \caption{\JointHeuristic, available at
    \url{https://github.com/tdhock/PeakSegJoint}}
\end{algorithmic}\label{algo}
\end{algorithm}

For example if we fix the zoom factor at
$\beta=2$, a demonstration of the algorithm on a small data set 
with $B=24$ points is shown in
Figure~\ref{fig:heuristic-algo}. \textsc{MaxBinSize} returns 4, so
\textsc{GridSearch} considers 15 models of $b=7$ data points at bin
size 4, and then \textsc{SearchNearPeak} considers 16 models each at
bin sizes 2 and 1. In the real data set of
Figure~\ref{fig:PeakSegJoint}, there are $B=85846$ data points,
\textsc{MaxBinSize} returns 16384, \textsc{GridSearch} considers 10
models of $b=6$ data points, and then \textsc{SearchNearPeak}
considers 16 models each at bin sizes 8192, 4096, ..., 4, 2, 1.

Overall \JointHeuristic\ with zoom factor $\beta$ searches a total of
$O(\beta^2\log B)$ models, and computing the likelihood and
feasibility for each model is an $O(pB)$ operation, so the algorithm
has a time complexity of $O(\beta^2 pB\log B)$. Thus for a count data
matrix $\mathbf Z\in\ZZ_+^{B\times S}$, computing the sequence of
\ref{PeakSegJoint} models $\mathbf{\hat M}^0(\mathbf Z), \dots,
\mathbf{\hat M}^S(\mathbf Z)$ takes $O(\beta^2 S B\log B)$ time.

Finally note that when there is only $S=1$ sample, \JointHeuristic\
can be used to find an approximate solution to the \ref{PeakSeg} model
with $p=1$ peak.

\subsection{Convex optimization for supervised penalty learning}

The convex supervised learning problem (\ref{argmin_w}) can be
solved using gradient-based methods such as FISTA, a Fast Iterative
Shinkage-Thresholding Algorithm \citep{fista}. 
% To apply FISTA, we need
% to compute the gradient of the smooth average surrogate loss
% \begin{equation}
%   \label{eq:average_gradient}
%   \nabla \mathcal L(\mathbf w) = 
%   \frac 1 n
%   \sum_{i=1}^n 
%   \nabla \ell \left[
%     \mathbf y_i,\,
%     f_{\mathbf w}(  \mathbf X_i )
%   \right],
% \end{equation}
% where the gradient of one observation $i$ is
% \begin{equation}
%   \label{eq:one_gradient}
%   \nabla \ell \left[
%     \mathbf y_i,\,
%     f_{\mathbf w}( \mathbf X_i )
%   \right]
%   =
%   \mathbf X_i \mathbf 1_S
%   \left[
%     \phi'\big(
%     f_{\mathbf w}( \mathbf X_i ) - \underline y_i
%     \big)
%     -
%     \phi'\big(
%     \overline y_i - f_{\mathbf w}( \mathbf X_i )
%     \big)
%   \right],
% \end{equation}
% and the derivative of the squared hinge loss is $\phi'(x)=2(x-1)I(x\leq 1)$.
For FISTA with constant step size we need a Lipschitz constant of
$\mathcal L(\mathbf w)$. Following the arguments of
\citet{hingeSquareFISTA}, we derived a Lipschitz constant of
$\sum_{i=1}^n ||\mathbf X_i||_F^2/n$. Finally, we used the
subdifferential stopping criterion of \citet{HOCKING-penalties}.

% An optimality condition for (\ref{argmin_w}) is 
% \begin{equation}
%   \label{eq:optimality}
%   \nabla \mathcal L(\mathbf w) \in \gamma \partial ||\mathbf w||_1.
% \end{equation}

\section{Results}
\label{sec:results}

\subsection{Accuracy of PeakSegJoint on benchmark data sets}

We used the 7 benchmark data sets of
\citet{hocking2014visual}, which included a total of 12,826 manually
annotated
region labels.\footnote{\url{http://cbio.ensmp.fr/~thocking/chip-seq-chunk-db/}}
Each data set contains labels grouped into windows of nearby regions
(from 4 to 30 windows per data set). For each data set, we performed 6
random splits of windows into half train, half test. Since there are
multiple peaks per window, and \ref{PeakSegJoint} can detect at most 1
peak per sample (\ref{zero_or_one}), we divided each window into separate
\ref{PeakSegJoint} problems of size $B$ (a hyper-parameter tuned via
grid search).

We compared the test error of \ref{PeakSegJoint} with three
single-sample models (Figure~\ref{fig:test-error-dots}). In all 7 of
the benchmark data sets, our proposed \ref{PeakSegJoint} model
achieved test error rates comparable to the previous state-of-the-art
\ref{PeakSeg} model. In contrast, the baseline hmcan.broad
\citep{HMCan} and macs \citep{MACS} models from the bioinformatics
literature were each only effective for a single experiment type
(either H3K36me3 or H3K4me3, but not both).  

% We attempted to compare with joint peak detectors from the
% bioinformatics literature. In particular, we considered the
% multi-sample JAMM and PePr algorithms \citep{JAMM, PePr}. The 
% benchmark data sets that we analyzed contain samples of 3 or more
% different cell types, but these algorithms are only meant for one or
% two cell types. Upon recommendation from the authors we attempted to
% run these algorithms independently on each cell type. For both
% algorithms we ran into problems. PePr stopped with an error when
% analyzing only one sample (at least two samples is necessary). For
% JAMM we also encountered an error. In conclusion, we found that JAMM
% and PePr were unable to analyze the  benchmark data sets.

\subsection{Speed of \JointHeuristic\ algorithm for PeakSegJoint}

\begin{figure}[b!]
  \centering
  \includegraphics[width=\textwidth]{figure-test-error-dots.pdf}
  \vskip -0.5cm
  \caption{Test error of peak detectors in the 7  
    benchmark data sets (panels from left to right). Each dot shows
    one of 6 train/test splits, and the black vertical line marks the
    mean of the \ref{PeakSegJoint} model.}
  \label{fig:test-error-dots}
\end{figure}

The \JointHeuristic\ algorithm is orders of magnitude faster than
existing Poisson segmentation algorithms
(Figure~\ref{fig:timings}). For simulated single-sample, single-peak
data sets of size $B\in\{10^1, \dots, 10^6\}$, we compared
PeakSegJoint with the $O(B\log B)$ unconstrained pruned dynamic
programming algorithm (pDPA) of \citet{Segmentor} (R package
Segmentor3IsBack), and the $O(B^2)$ constrained dynamic programming
algorithm (cDPA) of \citet{HOCKING-PeakSeg} (R package PeakSegDP). We
set the maximum number of segments to 3 in the pDPA and cDPA, meaning
one peak. Figure~\ref{fig:timings} indicates that the \JointHeuristic\
algorithm of \ref{PeakSegJoint} enjoys the same $O(B \log B)$
asymptotic behavior as the pDPA. However the \JointHeuristic\
algorithm is faster than both the pDPA and \ref{PeakSeg} cDPA by at
least two orders of magnitude for the range of data sizes that is
typical for real data ($10^2 < B < 10^6$).

These speed differences are magnified when applying these Poisson
segmentation models to real ChIP-seq data sets. For example, the hg19
human genome assembly has about $3\times 10^9$ bases. For the
H3K36me3\_AM\_immune data set, a problem size of about
$B=\text{200,000}$ bases was optimal, so running PeakSegJoint on the
entire human genome means solving about 15,000 segmentation
problems. PeakSegJoint takes about 0.1 seconds to solve each of those
problems (Figure~\ref{fig:timings}), meaning a total computation time
of about 25 minutes. In contrast, using the pDPA or \ref{PeakSeg} cDPA
would be orders of magnitude slower (hours or days of computation).
Finally, it is difficult to perform fair speed comparisons with the
unsupervised macs and hmcan.broad algorithms, since they can only be
trained using grid search of their %10--20 
hyper-parameters.

\section{Discussion}
\label{sec:discussion}

\subsection{Multi-sample PeakSegJoint versus single-sample PeakSeg}

The \ref{PeakSegJoint} model is explicitly designed for supervised,
multi-sample peak detection problems such as the benchmark data
sets that we considered. We observed that the single-sample
\ref{PeakSeg} model is as accurate as \ref{PeakSegJoint} in these data
(Figure~\ref{fig:test-error-dots}). However, any single-sample model
is qualitatively inferior to a multi-sample model since it can not
predict overlapping peaks at the exact same positions. Furthermore,
the $O(B \log B)$ \JointHeuristic\ algorithm can compute the
\ref{PeakSegJoint} model much faster than the $O(B^2)$ cDPA of
\ref{PeakSeg} (Figure~\ref{fig:timings}).

\subsection{PeakSegJoint versus other joint peak detectors}

We attempted to compare with the multi-sample JAMM and PePr
algorithms, two joint peak detectors from the bioinformatics
literature \citep{JAMM, PePr}. The \ref{PeakSegJoint} model is more
interpretable than these existing methods, since it is able to handle
several sample types. This was advantageous in data sets such as
Figure~\ref{fig:good-bad}, which contains 3 cell types: tcell, bcell,
and monocyte. The PeakSegJoint model can be run once on all cell
types, and the resulting model can be easily interpreted to find the
differences, since peaks occur in the exact same locations across
samples. In contrast, existing methods such as JAMM or PePr are less
suitable to analyze this data set since they are limited to modeling
only 1 or 2 cell types. We nevertheless downloaded and ran the
algorithms separately on each sample type, but both JAMM and PePr
produced errors and so were unable to analyze the benchmark
data sets.

\section{Conclusions}
\label{sec:conclusions}

We proposed the \ref{PeakSegJoint} model for supervised joint peak
detection. It generalizes the state-of-the-art \ref{PeakSeg} model to
multiple samples. We proposed the \JointHeuristic\ algorithm and
showed that it is orders of magnitude faster than existing Poisson
segmentation algorithms. Finally, we showed that choosing the number
of peaks in \ref{PeakSegJoint} using supervised penalty learning
yields test error rates that are comparable to the previous
state-of-the-art \ref{PeakSeg} model.

For future work, we would be interested in a theoretical analysis
analogous to the work of \citet{cleynen2013segmentation}, which could
suggest the form of the optimal penalty function to choose the number
of peaks in the \ref{PeakSegJoint} model.

\begin{figure}[b!]
  \centering
  \input{figure-timings}
  \vskip -0.5cm
  \caption{Timings of three Poisson segmentation algorithms on
    simulated data sets of varying size $B$. The grey shaded area
    represents the range of problem sizes selected in the 
     benchmark data sets. }
  \label{fig:timings}
\end{figure}

\newpage

\bibliographystyle{abbrvnat}

\bibliography{refs}

\end{document}
