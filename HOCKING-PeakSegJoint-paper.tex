\documentclass{article} % For LaTeX2e
\usepackage{nips15submit_e,times}
\usepackage{hyperref}
\usepackage{url}
\usepackage{natbib}


\title{PeakSegJoint: supervised detection of the same peaks jointly
  across several ChIP-seq samples}


\author{
David S.~Hippocampus\thanks{ Use footnote for providing further information
about author (webpage, alternative address)---\emph{not} for acknowledging
funding agencies.} \\
Department of Computer Science\\
Cranberry-Lemon University\\
Pittsburgh, PA 15213 \\
\texttt{hippo@cs.cranberry-lemon.edu} \\
\And
Coauthor \\
Affiliation \\
Address \\
\texttt{email} \\
\AND
Coauthor \\
Affiliation \\
Address \\
\texttt{email} \\
\And
Coauthor \\
Affiliation \\
Address \\
\texttt{email} \\
\And
Coauthor \\
Affiliation \\
Address \\
\texttt{email} \\
(if needed)\\
}

% The \author macro works with any number of authors. There are two commands
% used to separate the names and addresses of multiple authors: \And and \AND.
%
% Using \And between authors leaves it to \LaTeX{} to determine where to break
% the lines. Using \AND forces a linebreak at that point. So, if \LaTeX{}
% puts 3 of 4 authors names on the first line, and the last on the second
% line, try using \AND instead of \And before the third author name.

\newcommand{\fix}{\marginpar{FIX}}
\newcommand{\new}{\marginpar{NEW}}

%\nipsfinalcopy % Uncomment for camera-ready version

\begin{document}


\maketitle

\begin{abstract}
  Peak detection is a central problem in genomic data analysis, and
  current algorithms for this task are unsupervised and mostly
  effective for a single data type and pattern (e.g. H3K4me3 data with
  a sharp peak pattern). We propose PeakSeg, a new constrained maximum
  likelihood segmentation model for peak detection with an efficient
  inference algorithm: constrained dynamic programming. We investigate
  un\-super\-vised and super\-vised learning of penalties for
  the critical model selection problem. We show that the 
  super\-vised method has state-of-the-art peak
  detection across all data sets in a benchmark that includes both
  sharp H3K4me3 and broad H3K36me3 patterns.
\end{abstract}

\section{Introduction}

\subsection{Peak detection in ChIP-seq data}

Chromatin immunoprecipitation sequencing (ChIP-seq) is a biological
experiment for genome-wide profiling of histone modifications and
transcription factor binding sites, with many experimental and
computational steps \citep{practical}. Briefly, each experiment yields
a set of sequence reads which are aligned to a reference genome, and
then the number of aligned reads are counted at each genomic position
(Figure~\ref{fig:good-bad}). These data can be interpreted using
one of the many available peak detection algorithms
\citep{evaluation2010, rye2010manually, chip-seq-bench}, which each
essentially work as a binary classifier for each genomic position. The
positive class is enriched (peaks) and the negative class is
background noise. Importantly, peaks and background occur in long
contiguous segments across the genome.


\bibliographystyle{abbrvnat}
\bibliography{refs}

\end{document}
