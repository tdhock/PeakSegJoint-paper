% -*- compile-command: "make HOCKING-PeakSegJoint-slides.pdf" -*-
\documentclass{beamer}
\usepackage{tikz}
\usepackage[all]{xy}
\usepackage{amsmath,amssymb}
\usepackage{hyperref}
\usepackage{graphicx}

\DeclareMathOperator*{\argmin}{arg\,min}
\DeclareMathOperator*{\Lik}{Lik}
\DeclareMathOperator*{\Peaks}{Peaks}
\DeclareMathOperator*{\Segments}{Segments}
\DeclareMathOperator*{\argmax}{arg\,max}
\DeclareMathOperator*{\maximize}{maximize}
\DeclareMathOperator*{\minimize}{minimize}
\newcommand{\sign}{\operatorname{sign}}
\newcommand{\RR}{\mathbb R}
\newcommand{\ZZ}{\mathbb Z}
\newcommand{\NN}{\mathbb N}

% Set transparency of non-highlighted sections in the table of
% contents slide.
\setbeamertemplate{section in toc shaded}[default][100]
\AtBeginSection[]
{
  \setbeamercolor{section in toc}{fg=red} 
  \setbeamercolor{section in toc shaded}{fg=black} 
  \begin{frame}
    \tableofcontents[currentsection]
  \end{frame}
}

\begin{document}

\title{Supervised detection of the same peaks jointly across 
  several ChIP-seq samples}

\author{
  Toby Dylan Hocking\\
  toby.hocking@mail.mcgill.ca\\
  joint work with Guillem Rigaill and Guillaume Bourque}

\date{2 April 2015}

\maketitle

\input{figure-profiles}

\begin{frame}
  \frametitle{Comparison of PeakSeg and Joint model}

  \includegraphics[width=\textwidth]{figure-timings-profiles}
\end{frame}

\begin{frame}
  \frametitle{Comparing timings to segment 8 samples}

  \small

  Segment each sample (PeakSeg):

  \input{table-timings-PeakSeg}

  \vskip 0.2 cm

  Segment each genomic region (PeakSegJoint):

  \input{table-timings}

\end{frame}

\end{document}
